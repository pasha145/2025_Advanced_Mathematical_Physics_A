\documentclass[a4paper,11pt]{article}
\usepackage[utf8]{inputenc}
\usepackage{fullpage}
\usepackage{amsmath,amssymb,amsfonts}
\usepackage{tikz}
%\usepackage{hyperref}
%\usepackage{color}

\title{}
\author{}
\date{}

\newcommand{\tr}{\operatorname{tr}}
\newcommand{\diag}{\operatorname{diag}}
\newcommand{\Res}{\mathop{Res}}
%\numberwithin{equation}{section}

\begin{document}
%\maketitle

\begin{enumerate}
\item\label{item:1} Find the normal form of the equation \[u_{xx}+2 u_{xy}+u_{yy}+u_x+u=0.\]

\item\label{item:2} Find the normal form of the equation \[u_{xx}+2u_{xy}+u_{yy}-u_{zz}=0.\] Write its general solution.

\item\label{item:3} Check that the general solution of the equation \[\partial_x \left(\frac{x y}{x^2+y^2}\partial_x u\right)+\frac12 \partial_x \left(\frac{x^2-y^2}{x^2+y^2}\partial_y u\right)+\frac12 \partial_y \left(\frac{x^2-y^2}{x^2+y^2}\partial_x u\right) - \partial_y\left(\frac{x y}{x^2+y^2}\partial_y u\right)=0\]
is given by
\[u(x,y)=F(x^2-y^2)+G(x y).\]

\item\label{item:4} Solve the wave equation on the half-line \(x>0\):
\[u_{tt}-u_{xx}=\theta_H(1-x), \quad u(0,x)=0, \quad u_t(0,x)=0, \quad u(t,0)=0,\]
where
\(\theta_H\) is the Heaviside theta function:
\[\theta_H(x)=\left\{
\begin{array}{l}0, \quad x<0,\\
1, \quad x>0,\\
\frac12, \quad x=0.
\end{array}
\right.
\]
For simplicity, you may consider its solution only for \(t>4\).

\item\label{item:5} Solve the wave equation on the half-line \(x>0\):
\[
u_{tt}-u_{xx}=0, \quad u(0,x)=0, \quad u_t(0,x)=\theta_H(x-3)-\theta_H(x-2), \quad u(t,0)=0.
\]

For simplicity, you may again consider only the large time case.

\item\label{item:6} Solve the wave equation on the half-line:
\[
u_{tt}-u_{xx}=0, \quad u(0,x)=0, \quad u_t(0,x)=0, \quad u(t,0)=\sin t.
\]

\item\label{item:7} Compute the Fourier transform of the function \(f\) on \(\mathbb{Z}/(2N) \mathbb{Z}\) defined by
\[
f(0)=0, \quad f(1)=\ldots =f(N-1)=1, \quad f(N)=0, \quad f(-1)=\ldots =f(1-N)=-1.
\]

\item\label{item:8} Compute the Fourier transform of the  function \(f(\phi)\) on \(S^1\) defined by
\[
f(\phi)=\left[\begin{aligned}-1, & \quad \phi\in(-\pi,0)\\1, & \quad \phi\in(0,\pi)\end{aligned}\right.
\]

\item\label{item:9} Compute the [inverse] Fourier transform of
\[
\tilde{f}_{\epsilon}(p)=e^{-\epsilon|p|}
\]
defined on the real line.

\item\label{item:10} Compute the Fourier transform of
\[
f_{\epsilon}(x)=\frac{x}{\pi(x^2+\epsilon^2)}
\]
defined on the real line.
Compute the integral
\[
\int_{-\infty}^{\infty}f_{\epsilon}(x)dx
\]

\item\label{item:11} Compute the Fourier transform of
\[
f(n)=e^{-\epsilon|n|}
\]
defined on \(\mathbb{Z}\).

\item\label{item:12} Compute the [inverse] Fourier transform of
\[
\tilde{f}_\epsilon(p)=e^{-\frac12\epsilon p^2}
\]
What is the value of its integral from \(-\infty\) to \(+\infty\)?

\item\label{item:13} Compute the Fourier transform of the function on \(S^1\)
\[
f_{\epsilon}(\phi) = \frac{\sinh \epsilon}{\cosh \epsilon-\cos \phi}.
\]
Hint: for computing the integral \(\int_0^{2\pi} \frac{d\phi}{2\pi} f_{\epsilon}(\phi)e^{-i n \phi}\), introduce the new variable \(z=e^{i \phi}\) and find an appropriate way to deform the integration contour.

What is \(\int_0^{2\pi} f_{\epsilon}(\phi)d\phi\)?

\item\label{item:14} Solve the heat equation
\[
u_t=u_{xx}, \quad u(0,x)=e^{-a x^2}
\]

\item\label{item:15} Compute the Fourier transform of
\[
f(x)=x e^{-x^2}
\]
and of
\[
f(x)=x^2 e^{-x^2}
\]
\end{enumerate}

\end{document}
